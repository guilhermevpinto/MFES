\begin{vdmpp}[breaklines=true]
class TestGame is subclass of MyTest

instance variables

 private game : Game;
 
operations

(*@
\label{test:9}
@*)
 public test : () ==> ()
 test() ==
 (
  testConstructor();
  testMove();
  testFourPlayers();
 );
 
(*@
\label{testConstructor:17}
@*)
 private testConstructor : () ==> ()
 testConstructor() ==
 (
  
  dcl
  game: Game := new Game(2),
  p2: Player := new Player(3, 10);
  
  game.switchPlayer();
  assertTrue(game.getCurrentPlayer() = 2);
  
  game.move(1, 17, game.getPlayer(2));
  assertTrue(game.currentPlayerWin());
  
  game.switchPlayer();
  assertTrue(game.getCurrentPlayer() = 1);
  assertTrue(game.currentPlayerWin() = false);
  
  game.addPlayer(p2);
  assertTrue(len game.getPlayers() = 3);
  
 );
 
(*@
\label{testMove:40}
@*)
 private testMove : () ==> ()
 testMove() ==
 (
  dcl 
  game: Game := new Game(2),
  p1: Player := game.getPlayer(1),
  p2: Player := game.getPlayer(2),
  moves: seq of Point;
  
  game.eraseOldPosition(new Point(1, 17));
  assertTrue(game.getBoard().board([1,17]) = <FREE>);
  
  -- normal move case
  game.move(1, 1, game.getPlayer(1));
  assertTrue(game.getBoard().board([1,1]) = <OCCUPIED> and p1.getPosition().getX() = 1 and p1.getPosition().getY() = 1);
  game.move(3, 3, game.getPlayer(2));
  assertTrue(game.getBoard().board([3,3]) = <OCCUPIED> and p2.getPosition().getX() = 3 and p2.getPosition().getY() = 3);
  
  -- special move cases
  
  -- player up + wall up
  game.move(5, 7, game.getPlayer(1));
  game.move(3, 7, game.getPlayer(2));
  moves := game.getPossibleMoves();
  assertTrue(len moves = 4);
  
  assertTrue(game.addWall(2, 7));
  moves := game.getPossibleMoves();
  assertTrue(len moves = 5);
  
  -- player left + wall left
  game.move(5, 9, game.getPlayer(1));
  game.move(5, 7, game.getPlayer(2));
  moves := game.getPossibleMoves();
  assertTrue(len moves = 4);
  
  assertTrue(game.addWall(5, 6));
  assertFalse(game.addWall(5, 6));
  moves := game.getPossibleMoves();
  assertTrue(len moves = 5);
  
  -- player right + wall right
  game.move(5, 7, game.getPlayer(1));
  game.move(5, 9, game.getPlayer(2));
  moves := game.getPossibleMoves();
  assertTrue(len moves = 3);
  
  assertTrue(game.addWall(5, 10));
  moves := game.getPossibleMoves();
  assertTrue(len moves = 4);
  
  -- player down + wall down
  game.move(3, 9, game.getPlayer(1));
  game.move(5, 9, game.getPlayer(2));
  moves := game.getPossibleMoves();
  assertTrue(len moves = 3);
  
  -- test horizontal diagonal
  game.move(5, 15, game.getPlayer(1));
  game.move(7, 15, game.getPlayer(2));
  moves := game.getPossibleMoves();
  assertTrue(len moves = 4);
  
  assertTrue(game.addWall(8, 15));
  moves := game.getPossibleMoves();
  assertTrue(len moves = 5);
  
  -- test checkUpMove
  game.move(7, 17, game.getPlayer(1));
  game.move(5, 17, game.getPlayer(2));
  moves := game.getPossibleMoves();
  assertTrue(len moves = 2);
  
 );
 
(*@
\label{testFourPlayers:115}
@*)
 private testFourPlayers : () ==> ()
 testFourPlayers() ==
 (
 
  dcl
  game: Game := new Game(4);
  
  game.switchPlayer();
  assertTrue(game.getCurrentPlayer() = 2);
  game.switchPlayer();
  assertTrue(game.getCurrentPlayer() = 3);
  game.switchPlayer();
  assertTrue(game.getCurrentPlayer() = 4);
  game.switchPlayer();
  assertTrue(game.getCurrentPlayer() = 1);
  game.switchPlayer();
  game.switchPlayer();
  game.switchPlayer();
  assertTrue(game.getCurrentPlayer() = 4);
  
  game.getPlayer(4).setTargetCol(1);
  assertTrue(game.currentPlayerWin() = false);
 
 );
 
end TestGame
\end{vdmpp}
\bigskip
\begin{longtable}{|l|r|r|r|}
\hline
Function or operation & Line & Coverage & Calls \\
\hline
\hline
\hyperref[test:9]{test} & 9&100.0\% & 11 \\
\hline
\hyperref[testConstructor:17]{testConstructor} & 17&100.0\% & 11 \\
\hline
\hyperref[testFourPlayers:115]{testFourPlayers} & 115&100.0\% & 11 \\
\hline
\hyperref[testMove:40]{testMove} & 40&100.0\% & 11 \\
\hline
\hline
TestGame.vdmpp & & 100.0\% & 44 \\
\hline
\end{longtable}

